\section {第2章}
##{問題}{prob-munsterberg}{カフェウォール錯視}
\VerbatimInput[numbers=left,firstnumber=1, numbersep=6pt]
{CH2/svg-munsterberg.svg}
##{問題}{prob-delboef}{デルブーフの錯視}
\VerbatimInput[numbers=left,firstnumber=1, numbersep=6pt]
{CH2/svg-delboef.svg}
##{問題}{prob-gradient-mach}{マッハバンド錯視}
\VerbatimInput[numbers=left,firstnumber=1, numbersep=6pt]
{CH2/svg-gradient-mach.svg}
##{問題}{prob-zavagno}{ザバーニョの錯視}
\VerbatimInput[numbers=left,firstnumber=1, numbersep=6pt]
{CH2/svg-zavagno.svg}
##{問題}{prob-Koffka}{コフカリング}
\VerbatimInput[numbers=left,firstnumber=1, numbersep=6pt]
{CH2/svg-Koffka.svg}
##{問題}{prob-diamond-grad}{ひし形を用いたクレイク・オブライエン効果}
\VerbatimInput[numbers=left,firstnumber=1, numbersep=6pt]
{CH2/svg-diamond-grad.svg}
##{問題}{prob-craik-obraien}{クレイク・オブライエン効果}
\VerbatimInput[numbers=left,firstnumber=1, numbersep=6pt]
{CH2/svg-craik-obraien.svg}
##{問題}{prob-pyramid}{ピラミッドの稜線}
\VerbatimInput[numbers=left,firstnumber=1, numbersep=6pt]
{CH2/svg-pyramid.svg}
\section {第3章}
##{問題}{prob-object-gradiation}{\relax }
\par \noindent 
svg-polygon5-gradiation
 \par 
##{問題}{prob-objects-gradiation}{まとまった図形をグラデーションで図形を塗りつぶす}
\VerbatimInput[numbers=left,firstnumber=1, numbersep=6pt]
{CH3/svg-polygon5-gradiation4.svg}
##{問題}{prob-shepard}{シェパードの錯視}
\VerbatimInput[numbers=left,firstnumber=1, numbersep=6pt]
{CH3/svg-shepard.svg}
##{問題}{prob-bourdon}{ブルドンの錯視}
\VerbatimInput[numbers=left,firstnumber=1, numbersep=6pt]
{CH3/svg-bourdon.svg}
##{問題}{prob-savigny}{ザビニの錯視}
\VerbatimInput[numbers=left,firstnumber=1, numbersep=6pt]
{CH3/svg-savigny.svg}
##{問題}{prob-hermann-twinkle}{輝くヘルマン格子}
\VerbatimInput[numbers=left,firstnumber=1, numbersep=6pt]
{CH3/svg-hermann-twinkle.svg}
##{問題}{prob-munsterberg-pattern1}{カフェウォール錯視--パターンで作成}
\VerbatimInput[numbers=left,firstnumber=1, numbersep=6pt]
{CH3/svg-munsterberg-pattern.svg}
##{問題}{prob-morgan-twist}{モーガンのねじれひも}
\VerbatimInput[numbers=left,firstnumber=1, numbersep=6pt]
{CH3/svg-morgan-twist.svg}
\section {第4章}
##{問題}{prob-svg-fick-animation}{フィックの錯視(アニメーション付き)}
\VerbatimInput[numbers=left,firstnumber=1, numbersep=6pt]
{CH4/svg-fick-animation.svg}
##{問題}{prob-judd}{ジャッドの錯視}
\VerbatimInput[numbers=left,firstnumber=1, numbersep=6pt]
{CH4/jud.svg}
##{問題}{prob-rect-with-scale}{長方形が横に伸びる}
\par {何も属性がない\Showattrib {g}は\Showattrib {scale}のアニメーションを付けるのに必要です。}
\VerbatimInput[numbers=left,firstnumber=1, numbersep=6pt]
{CH4/svg-rect-with-scale.svg}
##{問題}{prob-zavagno-animation}{ザバーニョの錯視(アニメーション付き)}
\VerbatimInput[numbers=left,firstnumber=1, numbersep=6pt]
{CH4/svg-zavagno-animation.svg}
##{問題}{prob-svg-mask2}{\texttt  {stop}にアニメーションを付ける}
\par {前のものと二つ並べて同じ動きになることを確認できるようにしています。}
\VerbatimInput[numbers=left,firstnumber=1, numbersep=6pt]
{CH4/svg-mask2.svg}
##{問題}{prob-svg-move-square}{正方形の移動}
\VerbatimInput[numbers=left,firstnumber=1, numbersep=6pt]
{CH4/svg-move-square.svg}
##{図}{munsterberg}{カフェウォール錯視にアニメーションをつける}
\VerbatimInput[numbers=left,firstnumber=1, numbersep=6pt]
{CH4/svg-munsterberg-animation.svg}
##{問題}{prob-moving-square}{ゆがんだ正方形}
\VerbatimInput[numbers=left,firstnumber=1, numbersep=6pt]
{CH4/svg-move-square-amim.svg}
\section {第6章}
##{問題}{prob-Prob-add-objects}{直線の色を変える}
\par {ここでは色が二つ選べるようにしています。}
\VerbatimInput[numbers=left,firstnumber=1, numbersep=6pt]
{CH7/svg-js-add-lines-color.svg}
##{問題}{prob-prob-fick-interactive}{フィックの錯視図形をインターラクティブに操作する}
\par {終了時に長さがどれ位になったかを表示するようにしています。}
\VerbatimInput[numbers=left,firstnumber=1, numbersep=6pt]
{CH7/svg-fick-interactive.svg}
##{問題}{prob-zollner-animation}{ツェルナーの錯視(アニメーション付き)}
\VerbatimInput[numbers=left,firstnumber=1, numbersep=6pt]
{CH7/svg-js-zollner-animation.svg}
##{問題}{prob-svg-js-add-lines-rev2}{画面に直線を追加する(関数群で書き直す)}
\VerbatimInput[numbers=left,firstnumber=1, numbersep=6pt]
{CH7/svg-js-add-lines-rev2.svg}
##{問題}{prob-neckress-ring}{ネックレスの糸}
\VerbatimInput[numbers=left,firstnumber=1, numbersep=6pt]
{CH7/svg-neckress-string.svg}
##{問題}{prob-svg-cycloid-animation-start-click}{サイクロイドを描く --- クリックでアニメーションを開始}
\VerbatimInput[numbers=left,firstnumber=1, numbersep=6pt]
{CH7/svg-cycloid-animation-start-click.svg}
##{問題}{prob-random}{色と大きさと位置をランダム決めた円を表示する}
\VerbatimInput[numbers=left,firstnumber=1, numbersep=6pt]
{CH7/random.svg}
