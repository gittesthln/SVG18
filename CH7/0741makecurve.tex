% -*- coding: utf-8 -*-
\subsection{任意の形の曲線を描く}\label{SVGwithJavascript}
図\ref{cycloid}は直線上を円が滑らずに回転していくとき、その円周上に固定
された点が描く軌跡です。この曲線は\keyitem{サイクロイド}とよばれています。

\ShowFig{0.6}{ht}{cycloid}{サイクロイド}{cycloid}
半径 $r$ の円周上に固定された点が回転する直線上に初めにあ
るとき、角 $\theta$ だけ円が回転したときの点の位置は
\begin{equation}
\left\{
\begin{array}{rcl}
 x &=& r(\theta-\sin\theta)\\
 y &=& r(1-\cos\theta)
\end{array}\label{cycloid-formula}
\right.
\end{equation}
で与えられます。サイクロイドを描くためにはこの式で計算した点
を結ぶ直線で近似することにします。これらの座標をそのままSVGの中に直接記
述するのは面倒なので、\JS のプログラムで計算してその結果を \Elm{path}の
属性 \AttribO{d}に設定することにします。

%\newpage\ \\[-3\baselineskip]
\SVGListN{サイクロイドを描く}{svg-cycloid}{svg-cycloid}
\begin{itemize}
 \item 描く図形は円が転がっていく直線(\Line{base})とサイクロイドの図形
       (\Line{path})です。
 \item 転がる直線の$y$座標は$0$になっています。
 \item \Line{R}で回転する円の半径を保持する変数の値を設定します。
 \item \Line{init}から\SVG がロード終了後に呼び出される関数を定義しています。
 \item \Line{initvar}で必要な変数の宣言をしています。サイクロイドの道の
       りのデータを格納する変数\texttt{d}は\texttt{"M"}で初期化して
       います。
 \item \Line{start}から\Line{end}でサイクロイドの点の座標を計算します。ここで
は$0^{\circ}$から$1^{\circ}$単位で$360^{\circ}$まで計算します。
角度の単位がラジアンでないのはこの後の
リスト\ref{svg-cycloid-animation}で回転する円
を表示するアニメーションが付いた図形をSVGの\texttt{rotate}を用いて描くた
めです。
 \item \Line{changetoRad}で角度の単位をラジアンに直します。\JS では円周
       率の値を\ElmJ{Math.PI}で利用できます。
 \item \Line{getX}と\Line{getY}で式(\ref{cycloid-formula})に基づいて点の
       位置を計算しています。
       \keyitem{正弦関数}($\sin{x}$)
       \IndexSet{sinx@$\protect\sin{x}$(正弦関数)}{}{}{}{}{}{}
       と\keyitem{余弦関数}($\cos{x}$)
       \IndexSet{cosx@$\protect\sin{x}$(余弦関数)}{}{}{}{}{}{}{}
			 は\JS ではそれぞれ\ElmJ{Math.sin(x)}と\ElmJ{Math.cos(x)}で利用できます。

       なお、ここではSVGの座標系の関係から$y$座標の値は符号を逆にしてい
       ます。
 \item \Line{appendPath}で今までに求めた道のりのデータの後に新しく得られ
       た点の座標を付け加えています。
 \item \Line{setpath}で道のりのデータを書き直しています。
\end{itemize}
\begin{Problem}\upshape
 リスト\ref{svg-cycloid}を参考にして正6角形を描くSVGファイルを作成しなさ
 い。
\end{Problem}