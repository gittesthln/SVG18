%-*- coding: utf-8 -*-
\iffalse
表\ref{MethodDOM}は DOM のメソッドのリストです。\texttt{document}だけ
に適用できるものと、すべての要素に適応できるものとがあります。

本書ではこれらのいくつかについて使い方を解説します。
\fi
{\setlength{\tabcolsep}{0.2em}
\begin{longtable}{|c|c|m{20em}|}
\caption{DOMのメソッド}\label{MethodDOM}
\\  \hline
メソッド名  & {使用可能要素}&
\hspace*{\fill}説{\hfill}明\hspace*{\fill}\rule{0em}{0em}\\ \hline
\endfirsthead
\caption{DOMのメソッド(続き)}
\\  \hline
メソッド名  & {使用可能要素}&
\hspace*{\fill}説{\hfill}明\hspace*{\fill}\rule{0em}{0em}\\ \hline
\endhead
\hline\multicolumn{3}{r}{次ページへ続く}
\endfoot
\endlastfoot
\DOMM{getElementById}{(id)}&\texttt{document}&
      属性\texttt{id}の値が引数\texttt{id}である要素を得る。 \\\hline
\DOMM{getElementsByTagName}{(Name)}&対象要素&
 対象要素の子要素で要素名が\texttt{Name}であるもののリストを得る。\\\hline
\DOMM{getElementsByClassName}{(Name)}&対象要素&
     属性\texttt{class}の値が\texttt{Name}である要素のリストを得る。 \\\hline
\DOMM{getElementsByName}{(Name)}&\texttt{document}&
     属性\texttt{name}が\texttt{Name}である要素のリストを得る。\\\hline
\DOMM{querySelector}{(selectors)}&対象要素&
     \texttt{selectors}で指定されたCSSのセレクタに該当する一番初めの要素
	  を得る。 \\\hline
\DOMM{querySelectorAll}{(selectors)}&対象要素&
     \texttt{selectors}で指定されたCSSのセレクタに該当する要素のリストを得る。
	  \\\hline
\DOMM{getAttribute}{(Attrib)}&対象要素&
     対象要素の属性\texttt{Attrib}の値を読み出す。得られる値はすべて文字
	  列である。\\ \hline
{\DOMM{setAttribute}{(Attrib,Val)}}  &対象要素&
     対象要素の属性\texttt{Attrib}の値を\texttt{Val}にする。数を渡しても
	  文字列に変換される。\\ \hline
{\DOMM{hasAttribute}{(Attrib)}}  &対象要素&
     対象要素に属性\texttt{Attrib}がある場合は\texttt{true}を、ない場合
 は\texttt{false}を返す。\\ \hline
{\DOMM{removeAttribute}{(Attrib)}}  &対象要素&
     対象要素の属性\texttt{Attrib}を削除する。\\ \hline
\DOMM{getNodeName}{()}&対象要素&対象要素の要素名を得る。\\\hline
\DOMM{createElement}{(Name)} &\texttt{document}&
     \texttt{Name}で指定した要素を作成する。 \\ \hline
\DOMM{createElementNS}{(NS,Name)} &\texttt{document}&
     \keyitem{名前空間}\texttt{NS}で定義されている要素\texttt{Name}を作成
	  する。 \\ \hline
\DOMM{createTextNode}{(text)} &\texttt{document}&
     \texttt{text}を持つテキストノードを作成する。\\ \hline
{\DOMM{cloneNode}{(bool)}} &対象要素&
%   対象要素の複製を作る。
\texttt{bool}が
  \texttt{true}のときは対象要素の子要素すべてを、%複製する。
  \texttt{false}のときは対象要素だけの複製を作る。\\ \hline
{\DOMM{appendChild}{(Elm)}} &対象要素&
  \texttt{Elm}を対象要素の最後の子要素として付け加える。\texttt{Elm}がすでに
	  対称要素の子要素のときは元の位置から最後の位置に移動する。 \\ \hline
{\DOMM{insertBefore}{(newElm, PElm)}} &対象要素&
   対象要素の子要素\texttt{PElm}の前に\texttt{newElm}を子要素として付け
  加える。\texttt{Elm}がすでに対称要素の子要素のときは元の位置から指定さ
	  れた位置に移動する。 \\ \hline
\DOMM{removeChild}{(Elm)} &対象要素& 対象要素の子要素
      \texttt{Elm}を取り除く。\\ \hline
\DOMM{replaceChild}{(NewElm, OldElm)} &対象要素& 対象要素に含まれる子要素
      \texttt{OldElm}を\texttt{NewElm}で置き換える。\\ \hline
\DOMM{setValue}{(value)} &\small テキストノード& {対象のテキストノードの値を
	  \texttt{value}にする。}\\ \hline
\end{longtable}
}
いくつか注意をします。
\begin{itemize}
	 \item 表中の名前空間(Namespace)とは、指定した要素が定義されている規格を指定するもの
です。一つの文書内で複数の規格を使用する場合、作成する要素がどこで定義
されているのかを指定します。これにより、
				 異なる規格で同じ要素名が定義されていてもそれらを区別することが
				 可能となります。
				 
				 \ref{ExchangeDatabetweenSVGandHTML}節ではHTMLの要素とSVGの要素
				 を同時に取り扱ういます。
 \item CSS セレクタをまとめたものは付録\ref{CSSdefs}を参照してください。
       \ifSeminor\else
			 具体例はリスト\ref{pinna-storage.js}(\pageref{pinna-storage.js}ペー
			 ジ)にあります。
       \fi
 \item 要素のリストが得られるメソッドの戻り値のの各要素は
	  配列と同様に\texttt{[\hspace{0.1em}]}で参照でできます。
\end{itemize}

