% -*- coding: utf-8 -*-
\subsection{\keyitem{\JS }}
\subsubsection{\JS とは}
\JS はHTML文書の中に書くことが
できるスクリプト言語です\footnote{\JS が使えるものとしてはこのほか
にpdfやsvgもあります。決してHTML専用ではありません。}。
\JS  については入門書がたくさんありますので歴史などについてはそち
らを参考にしてください。一通り \JS のプログラムが書けるようになったら
\appendixname \cite{JavaScript} で計算機言語としてもう一度復習するとよい
でしょう。

\JS で書かれたプログラムは HTML 文書や SVG 文書の中にあり、その処理はブラ
ウザで行われます。文書が読み込まれるときに文書自体を作成するために
\texttt{document.write}という方法でするのが一般に行われていますが、
ここでの解説ではマウスのクリックなどで表示を変えたり
する DOM(Document Object Model) の技法を用いて同等のことを行うことにしま
す。DOM の技法を取り扱う \JS のライブラリーもありますが、基本的なことを
理解するためにこれらのライブラリーはこのテキストでは使用しません。
この技法は最近話題となっている Ajax を利用するためにも欠かせないものです。
\iffalse
この観点から \JS ではセキュリティを保つために動作して
いるコンピュータ上のファイルを読み出したり、書き込むことはできま
せん(ファイルのアップロ-ドは例外です)。

クライアント側のデータを保存するためには一度サーバー側に保存したい
データを送り返してもらうしか方法がありません。
\fi

\JS はインタープリター型の言語です。言語を解釈しながら実行していきますの
で途中まで正しいプログラムを書いてあるとそこまでは実行され、そのあとエラー
が出て停止することもあります。最近のブラウザーでは\JS のデバッガーが付い
ていて、ブラウザー自身が Web アプリケーションの開発のプラットフォームに
なっています。プログラムが動かないときはこの機能を利用するのが一般的です。

Chromeでこの機能を使うには
アドレスバーの右端にある「$\vdots$」をクリックし、「その他のツール」
			$\Rightarrow$「デベロッパーツール」
を選択します。また、一般のブラウザではショートカットキーとして「Cntl+Shift+i」または
「F12」が利用できます。

具体的な使用例は次節以降で紹介します。


