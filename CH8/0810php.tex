% -*- coding: utf-8 -*-
%\section{サーバーとの簡単な交信で行う例}
\section{PHPプログラムでSVGファイルを作成する}
次のHTML文書は表題と表示する正$n$角形の$n$を入力するためのテキストボック
スがあるだけの簡単なものです(図\ref{npolygon-serverG})。

\ShowFig{0.4}{ht}{npolygon-server}{正$n$角形を書く}{npolygon-serverG}

このテキストボックスに数字を入れて(ここでは$6$を入力します。)
「実行」ボタンを押すと図\ref{npolygon-server-res}が表
示されます。
\ShowFig{0.4}{ht}{npolygon-server-res}
{正$n$角形を書く(結果)}{npolygon-server-res}

リスト\ref{npolygon-server}は図\ref{npolygon-serverG}のものです。
\HTMLListN{正$n$角形を書く}{npolygon-server}{npolygon-server}
このリストの解説はサーバとのやり取りをする部分だけにします。
\begin{itemize}
 \item \LineR{FormsS}{FormsE}でフォームを定義しています。
       \Line{FormsS}の\ElmH{form}の属性は次の意味を持ちます。
\begin{itemize}
 \item フォームにはデータの送り方を指定する\AttribH{method}を
       \AttribHVal{PUT}{}に指定しています。
 \item データを送るときに呼び出されるURLは\AttribH{action}で指定します。
       ここでは\texttt{./svg-polygon.php}が指定されているので同じサーバ
       にある\texttt{./svg-polygon.php}が呼び出されます。
\end{itemize}
 \item テキストボックスは\Line{textbox}で定義されています。
\iffalse
       入力を与える
       ものは\ElmH{input}です。ここの属性は次のような意味を持ちます。
\begin{itemize}
 \item 属性\AttribH{type}は短い文字列を入力するためのものです。属性
f       \AttribH{size}でその文字列の表示される長さの最大値を指定します。
 \item
      \fi
      属性\AttribH{name}はここで与えられるデータの名前になります。

       図\ref{npolygon-server-res}のアドレスバーが次のようになっています。
\begin{center}
 \texttt{http://localhost/svg-polygon.php?N=6}
\end{center}
       \AttribHVal{PUT}{}によるデータの送信では URL のあとの\texttt{?}以降はプ
       ログラムに渡される引数になります。このような形で引数が渡されると
       きはPHPのプログラムではスーパーグローバル変数\texttt{\$\_GET}の中
       に代入されます。
%\end{itemize}
 \item \Line{button}に定義されているボタンは\AttribH{submit}ボタンです。
       \AttribHVal{SUBMIT}{}のボタンが押されるとフォームの
       \AttribH{action}で指定されたURLへ移動します。ここではPHPプログラ
       ムが起動されます。
\end{itemize}

次に、呼び出されるPHPプログラムを見ることにします。本質的にsvgファイルを順
次表示するようになっています。はじめにSVGの要素を出力するための関数群か
ら見ていきます。
\PHPListN{SVGを出力するための関数}{svg-func}{svg-func.php}
PHPで関数を定義するためには\ElmP{function}を用います。

ここでは次の関数を定義しています。
\paragraph{\texttt{printHeader(\$width, \$height)}}
SVGファイルの先頭部分のXML宣言とルート要素の\Elm{svg}を出力します。
PHPでは変数名は\texttt{\$}ではじめます。
\ListSub{printHeaderS}{printHeaderE}
\begin{itemize}
 \item 引数\texttt{\$width}と\texttt{\$height}は
       それぞれ\Elm{svg}内の属性\AttribO{width}と\AttribO{height}の値にな
       ります。
 \item 3行目から\Line{hereDocumentE}の部分は
       \keyitem{ヒアドキュメント}
       \ElmP{ヒアドキュメント}と呼ばれる方
       法で文字列を定義して出力しています。
\begin{itemize}
\item  \ElmP{<<<}がヒアドキュメントの開始を示しています。
 \item  この直後の文字列(ここでは
       \texttt{\_EOF\_})が行の先頭に来るまでがここでの文字列の範囲です
       (ここでは11行目に現れます)。最後に文の終了を示すセミコロンを忘れ
	ないようにしましょう。また、行の先頭に空白を入れてはいけません。
 \item この文字列に含まれる変数の部分は変数の値に置き換えられます
       \footnote{PHPでは文字列を\texttt{'}ではさむ場合は置き換え
       られません。改行を示す\texttt{\textbackslash n}も\texttt{'}ではさ
       まれたときはそのままの文字になります。}。
       この例では\texttt{\$height}と\texttt{\$width}がその値に置き換えられます。
\end{itemize}
 \item この文字列を\ElmP{print}を用いて標準出力に出力します。
 \item 最後にルート要素である\Elm{svg}をグローバル変数\texttt{setTags}の配列
       に設定しています。PHPでは関数内の変数はすべてローカルです。トップ
       レベルの変数をグローバル変数として利用するためには
       {スーパーグローバル変数}\ElmP{\$GLOBAL}
\IndexSet{\protect\ElmP{\$GLOBAL}}{スーパーグローバル変数}{}{}{PHP}の
       連想配列\IndexSet{PHP}{連想配列}{E}{における}{}を利用する必要があります。
\end{itemize}       
\paragraph{\texttt{outputSpaces(\$n)}}
       出力する要素にインデントをつけるため、現在の要素の深さの2倍の空白を
       出力する関数です。ここでは与えられた文字列を与えられた回数だけ繰
       り返す文字列を作成する関数\ElmP{str\_repeat()}を用いています。
\ListSub{outputSpacesS}{outputSpacesE}
\paragraph{\texttt{outputTag(\$tagName, \$attributes, \$close=false)}}
       引数\texttt{\$tagName}で与えられた要素名と引数\texttt{\$attributes}
       で与えられた属性値をもつ要素の開始部分を出力します。
\ListSub{outputTagS}{outputTagE}
\begin{itemize}
 \item 出力を見やすくするために要素の先頭に空白を入れます。
       その数はグローバル変数\texttt{setTags}の
       配列の大きさを\ElmP{count()}関数で求めています
       (\Line{calloutputSpaces0})。
 \item 次に与えられた要素の開始部分を出力します(\Line{printOpenTag})。
 \item 次に{連想配列}\texttt{\$attributes}で与えら
       れた要素の属性と属性値を取り出して
\begin{center}
 属性\texttt{="}値\texttt{"}
\end{center}
の形で出力します(\Line{printAttribsS}から\Line{printAttribsE})。
 ここでは\ElmP{foreach}を用いた連想配列のすべての要素を取り出す手法を用い
       ています。\JS ではキーしか取り出せませんでしたが、PHP ではキーと
       その値を簡単に取り出すことができます。ここではもちろん、次の関係
       が成立しています。
\begin{center}
 \texttt{\$attributes[\$key] = \$value}
\end{center}
 \item ここで得られた値をもとに属性値と属性名を出力しています
       (\Line{outputKeyVal})。
       ここでは\texttt{"}%"
       を用いた文字列を定義しています。この形の文字列内の変数は変数の値
       に置き換えられます。文字列を表す方法としては\texttt{'}を用いる方
       法もありますが、この定義では変数や\texttt{\textbackslash n}などの
       制御コードも解釈されません(\Line{printCloseTag}参照)。
 \item 引数\texttt{\$close}は要素の終了を定義する(\ElmP{true})か
       しないか(\ElmP{false})与える引数です。要素の終了を定義する場合
       (\Line{CheckCloseS})には\texttt{/}を出力し(\Line{CheckClose1})、
       しない場合には与えられた要素名をグローバル変数\texttt{setTags}の
       配列の最後に付け加えます(\Line{pushTag})。
       このために配列の最後に要素を付け加える\ElmP{array\_push()}関数を
       利用します。後に出てくる\ElmP{array\_pop()}関数とあわせると
       スタックが簡単に実現できます。\IndexSet{実現する(PHP)}{スタック}{T}{を}{}

\texttt{\$close}の仮引数に値\ElmP{false}が指定してあります。これはこの関
       数が呼び出されるとき、この引数が明示的に書いていない場合に利用さ
       れる値を定義しています。

このようにPHPでは最後に現れるいくつかの引数を関数呼び出しのときに付け加
       えなくても自動的に補完する機能があります。

 \end{itemize}       
\paragraph{\texttt{closeTag()}}
       まだ閉じられていない最新のタグの要素の終了部分をひとつ出力します。
\ListSub{closeTagS}{closeTagE}
       出力
       された要素名はグローバル変数\texttt{setTags}の配列から取り除かれま
       す。\Line{printCloseTag}の\ElmP{array\_pop()}関数は与えられた配列
       の最後のデータを配列から取り除きます。与えられた配列の大きさはひ
       とつ小さくなります。
 \paragraph{\texttt{closeSVG()}}
       閉じられていない要素をすべて閉じるための関数です。
\ListSub{closeSVGS}{closeSVGE}
       グロ−バル変数
       \texttt{setTags}の配列の大きさが$0$になるまで\texttt{closeTag()}
       関数を呼んですべての要素を閉じます。


次のリストは図\ref{npolygon-serverG}で実行ボタンが押されたときに呼ばれる
PHPファイルです。
\PHPListN{正$n$角形を書く}{svg-polygon}{svg-polygon.php}
\begin{itemize}
 \item \Line{requireonece}でリスト\ref{svg-func.php}のファイルを
       \ElmP{require\_once()}関数を用いて読み込みます。この関数は実行する
       PHPファイルがあるフォルダー
       (ディレクトリ)を基準としたとこから引数で指定したファイルを読み込
       みます。 \ElmP{require\_once()}関数は名前からもわかるように同じファ
       イルを2回は読み込みません。
 \item \Line{drawPathS}から\Line{drawPathE}が正$n$角形の頂点の座標を計算
       して、文字列に直す関数\texttt{drawPath()}を定義している部分です。
\begin{itemize}
 \item \Line{GET}でフォームで入力された値を参照しています。フォーム
       の
       \ElmH{input}では\AttribH{name}の値が\texttt{N}で、データ送信の方法
       が\ElmP{PUT}なのでこの値は\verb+$_GET['N']+で参照できます。
       \ElmP{\$\_GET}
       この値を変数\verb+$N+に代入しています。
\IndexSet{\protect\ElmP{\$\_GET}}{スーパーグローバル変数}{}{}{PHP}
 \item \Line{initString}に現れる\texttt{MaxSize}は\Line{define}の
       \ElmP{define()}関数で\texttt{200}に定義しています。正多角形の頂点の
       開始位置は$(\textrm{MaxSize,0})$に設定しますのでこの値を文字列の
       初期値として設定しています。
 \item 残りの頂点の座標は$(\textrm{MaxSize}\cos 2\pi
       i/N,\textrm{MaxSize}\sin 2\pi i/N)$ 
       $(i=1,2,\ldots,N-1)$ です。
 \item $(i=1,2,\ldots,N-1)$の繰り返しは\ElmP{for}で行います
       (\Line{calcS})。\texttt{for(初期化;継続条件;後処理)}の形式で記述
       します。
 \item \Line{toRad}にある\ElmP{M\_PI}はPHPで定義されている
       \keysubT{円周率}{PHP}{における}
       を表す定数です。PHP
       では三角関数の値はC言語のように弧度法で計算します。SVGの回転の角
       度が60分法であったのと異なります。ここではこの値を2度使いますので
       変数\verb+$Angle+にしまっています。
  \item \Line{toString}で計
       算して\texttt{pathString}に付け加えています。数値を文字列に直すた
       めに\ElmP{sprintf()}を用いています。変換の文字列の\Verb+%.4f+は小数
       点以下を4桁に設定することを指示しています。

       \ElmP{sprintf()}関数の戻り値は変換後の文字列です\footnote{C言語の
		\texttt{sprintf}関数の第1番目の引数は結果を格納する文字の配列を
	指定します。}。

 \item \Line{calloutputTagS}から\Line{calloutputTagE}で先ほど定義した
       \texttt{outputTag}関数を用いて\Elm{polygon}を出力します。ここでは
       引数を2つしか与えていないのでこの要素はこの時点で閉じられていま
       せん。
 \item この関数の2番目の引数は属性名をキーに値を属性値にとる連想配列の形
       です。このために\ElmP{array()}関数で配列を作成しています。
\end{itemize}
 \item ブラウザからサーバーにアクセスするということはサーバーにファイ
       ルの転送を依頼することと同じです。ファイルを転送する前にこれから
       送られるファイルがどのような種類のものなのかをサーバーがブラウザ
       に知らせます。この処理をするのが\Line{header}にある\ElmP{header()}
       関数です(\Line{header})。 
 \item ここでは送られるファイルの種類が画像(\texttt{image})で形式が
       \texttt{svg+xml}であることを知らせています\footnote{
       データの意味と形式を表す方この法は\keyitem{MIME(Multipurpose
       Internet Mail Extensions)}と呼ばれます。}。この関数をプログラム
       からよばないときには通常の設定ではPHPが\texttt{text/html}を送ります。
       \ElmP{header()}関数はこれ以外にも転送するファイルの大きさや別のペー
       ジに飛ばすことができたりします。
 \item \Line{startSVG}ではsvgファイルの先頭に書くべきものを作成しています。
 \item \Line{outputG}では\Elm{g}を、\Line{callPolygon}では正$n$角形のデー
       タを、\Line{closeTag}ではすべての閉じられていない要素の終了を
       すべて書くことを指示しています。
\end{itemize}
この実行結果のSVGファイルのリストは次の通りです。\Attrib{points}の部分は一行が長いので
改行しています。
\begin{Verbatim}
<?xml version="1.0" encoding="utf-8" ?>
   <svg xmlns="http://www.w3.org/2000/svg"
        xmlns:xlink="http://www.w3.org/1999/xlink"
        height="100%" width="100%">
   <title>正n角形の表示</title>  <g transform="translate(210,210)" >
    <polygon points="200,0 100.0000,173.2051 -100.0000,173.2051
                    -200.0000,0.0000 -100.0000,-173.2051 
                     100.0000,-173.2051"
             fill="blue" stroke-width="4" stroke="red" >
    </polygon>
  </g>
</svg>
\end{Verbatim}
\Elm{polygon}に終了要素があることに注意してください。
\begin{Problem}\upshape
リスト\ref{svg-polygon.php}について次の問いに答えなさい。
\begin{enumerate}
 \item \texttt{</polygon>}を出力しないようにしなさい。
 \item 多角形が表示されたときのアドレスバーの内容を確認しなさい。また、
       その内容を直接変更してアクセスしたときに何が起こるか確認しなさい。
\end{enumerate}
\end{Problem}

\section{\texttt{header}関数について}
ブラウザからあるサーバーにアクセスする場合、データのやり取りは目に見え
るものだけではありません。データのやり取りをする前にこれからどのような情
報が必要なのか、アクセスする側の情報はなにかなどをサーバーに伝えます。ま
た、サーバーからも必要なデータが来る前に、アクセスの結果や送られるファイ
ルの情報などを前もって送ります。

このような情報を積極的に伝えるために\ElmP{header()}関数があります。すでに
見てきたように\ElmP{header()}では次のような情報が送られてきました。
\begin{itemize}
 \item \texttt{Content-Type} これから送られるファイルの形式(メディアタイ
       プ)
 \item \texttt{Content-Length} これから送られるファイルの大きさ。\\ファイ
       ルをダウンロードするときにファイルの大きさが示される場合にはこの
       項目が送られてきています。PHPなどでその場で作成される文書は大きさ
       がわからないので送られないこともあります。
\end{itemize}
PHPのプログラムでははじめになにかデータをクライアント側に送るとデフォル
トの情報が送られてしまいます。その後で\ElmP{header()}関数を用いると「すで
に\ElmP{header()}が送られた」といエラーが発生します。たとえば、PHPのプロ
グラムの開始を示す\texttt{<?php}の\texttt{<}の前に空白があるとそれだけで
データの送付が開始されたということになります。

また、Windowに付属のメモ帳ではファイルの文字コードをUTF-8で保存するとファイル
の先頭に\keyitem{BOM(Byte Order Mark)}と呼ばれる2バイトの文字列が付け加
わり、これがデータとして解釈
されて\ElmP{header()}関数がうまく作動しない場合があります。

いくつかのエディタでは保存の形式で \keyitem{UTF-8} と \keyitem{UTF-8N} の
選択ができるものがあります。後者はBOMをつけない形式のようです。

Web のページをアクセスするとよくファイルのダウンロードのメッセージボック
スが開く場合があり、保存場所や送られてきたファイル名を変えて保存ができる
ようになっている場合があります。このようなことも\ElmP{header()}関数を用
いることで実現できます。

リスト\ref{svg-polygon.php}の\Line{header}の後に次の行を付け加えてくださ
い。
\begin{Verbatim}
header('Content-Disposition: attachment; filename="'.$_GET['N'].'-polygon.svg"');
\end{Verbatim}
前半の部分がファイルのダウンロードを指示していて、\texttt{filename}以降
の部分でファイル名を指定しています。\texttt{N=86}で実行すると図
\ref{npolygon-server-res2}のようにブラウザの下にファイルがだダウンロード
されます。
\ShowFig{0.4}{ht}{npolygon-server-restoFile}
{正$n$角形を書く(ファイルに保存)}{npolygon-server-res2}

このファイルの内容が\ElmP{print}により、クライアント側に送られます。
現存するファイルを送りたいときは\Line{header}内のMIMEタイプを送りたいファ
イルに合うように直し(たとえば、PDFファイルならば\texttt{application/pdf}
となります)、そのあとに\ElmP{readfile()}関数を用いてファイルを送ります。
\begin{Problem}\upshape
次の事項に答えなさい。
\begin{enumerate}
 \item BOM が存在するための理由
 \item Windows のメモ帳でUTF-8で保存したファイルと 他のエディタで UTF-8N
       で保存した同じファイルのファイルの大きさを確認すること
\end{enumerate}
 
\end{Problem}
\section{ブラウザを区別する}
クライアント側とサーバーの間ではユーザが目に見えないところでいろいろなや
り取りをしています。PHPを用いるとそれらのデータの一部にアクセスすることが可能
です。

リスト\ref{server}はPHPの\ElmP{\$\_SERVER}
\IndexSet{\protect\ElmP{\$\_SERVER}}{スーパーグローバル変数}{}{}{PHP}
変数で渡される情報の一覧を出力するものです。

\PHPListN{サーバに伝えられる情報を調べる}{server}{server}

ここではこのプログラムの出力をセキュリティ上の観点から
すべて見せるわけにはいきま
せん。この出力の中には次のような行があるはずです。
\begin{Verbatim}
HTTP_USER_AGENT	Mozilla/5.0 (Windows NT 6.2; Win64; x64)
  AppleWebKit/537.36 (KHTML, like Gecko) Chrome/56.0.2924.87 Safari/537.36
\end{Verbatim}
\ElmP{HTTP\_USER\_AGENT}はアクセスしたユーザのブラウザに関する情報が示され
ます(正確にはブラウザが伝えたブラウザ情報です)。これによりブラウザは
Chrome であり、使用OSはWindows NT 6.2であることがわかります。

\iffalse
なお、Windows のコマンドプロンプトで \texttt{winver}と打つとWindows の公
式のバージョンを表示させることができます(図\ref{WindowsVersion})。
%
\ShowFig{0.4}{ht}{WindowsVersion}
{Windows のバージョン}{WindowsVersion}

上の値がここで示されたバージョン番号とあっていることを確認してください
\footnote{このファイルを作成したパソコンのOSは Windows 8 ですがバージョ
ンがNT6.2となっていてメジャー番号がVistaと同じす。
\texttt{http://ja.wikipedia.org/wiki/Microsoft\textunderscore
Windows\textunderscore 10}によると Windows
10 ではメジャー番号が10になるようです。}。
\fi
したがってこれらの情報によりプログラムで返すファイルの内容を変える
ことが可能となります。
\begin{Problem}\upshape
 いろいろブラウザやOSを変えて\verb+$_SERVER['HTTP_USER_AGENT']+がどの
 ような値になるか調べなさい。特に、携帯電話のフルブラウザに対して調べる
 こと。\footnote{\texttt{http://www.hilano.org/s.php}にアクセ
 スするとこの値が返されます。}
\end{Problem}
