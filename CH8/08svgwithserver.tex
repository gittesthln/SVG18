% -*- coding: utf-8 -*-
\chapter{WebサーバーでSVG図形のデータを作る}\label{svgwithserver}
前章では\JS を用いていろいろなSVGの図形を操作する方法を学びました。この
章ではブラウザから与えられた情報をもとにWebサーバでSVGファイルを作成
したり、計算の一部を代行する方法を紹介します。
\section{Webサーバーを利用するための基礎知識}
この章のプログラムを実行するためにはWebサーバを必要とします。
手元にある PC に Web サーバーを設置して、その上で PHP プログラムを処理できる
ように設定する方法については
XAMPP\footnote{\texttt{https://www.apachefriends.org/jp/index.html}}をイ
ンストールするのが簡単です。パッケージをダウンロードして簡単にインストー
ルできます。
%付録\ref{InstallApachePHP}を参考にしてください。

Webページ上でデータを記入できたり、いくつかの選択肢からデータを選択した
りしてボタンを押すことでWebサーバにデータを送ることができるWebページは
フォームと呼ばれます。ここではHTML文書のフォームを通じて
データをサーバーに送り、
そのデータを下にSVGファイルを作成する方法を考えます。一般にこのような操
作は\keyitem{CGI}と呼ばれます。CGIプログラムは与えられたデータをもとに
HTML文書をその場で作成してユーザに返します。

近年ではHTML文書の一部のデータだけを返してもらい、そのデータをもとにペー
ジを書き換えるという\keyitem{Ajax}という技術もあります。Ajax で返
されたデータをもとにWebページを書き直す技術は前章で学んだ\keyitem{DOM}の
技術をHTML文書に応用します。この章ではこの技術の基本についても述べます。

ユーザから送られてきたデータを下に新しいデータを送り返す処理をするプログ
ラム言語としてこの章では\keyitem{PHP}を用います。
\Href{http://www.php.net/}{PHP}はhttpサーバーとして有名な
\Href{http://www.apache.org/}{Apache}\IndexSet{Apache}{}{}{}{}の中に組み
込んで動作するCGI言語です。
PHPはWebサーバとの連携を目的として言語設計がなされていますのでユーザから
送られたデータ受け渡しや、異なるHTML文書間のデータの受け渡しが比較的楽に
できます。

デバッグ用であればコマンドラインからPHPに\texttt{-S localhost:8080}のオプションを付
ければサーバーモードで起動させることもできます\footnote{この場合、httpの
サービスはポート番号が\texttt{8080}で起動されるのでファイルのアクセスは
\texttt{http://localhost:8080/...}となります。}。このときは起動したカレン
トディレクトリが\keyitem{ドキュメントルート}になります。

PHPはWebサーバに組み込まれて動作するだけ
ではなく単独で起動してデータを処理することも可能です\footnote{本書を作成するにあ
たり、索引の作成などにPHPプログラムを作成して処理させています。}。

PHPのプログラムはHTMLファイルの中の\texttt{<?php}と\texttt{?>}の間に
書きます。ファイルの拡張子は(たとえPHPのプログラムがまったくなくても
)\texttt{.php}が標準です。
\texttt{<?php}と\texttt{?>}の間の部分はWebサーバー上で処理され、クライア
ント側ではPHPのプログラムは見ることができません。

詳しい文法などについては%\cite{php10}や
\cite{learningphp}などをを参考にしてください。

%PHP のプログラムの簡単な紹介は付録\ref{php}を参考にしてください。
\iffalse\else
\input CH8/0810php.tex
%\input CH8/0825saveData.tex
\input CH8/0820Ajax.tex

%\newpage
\input CH8/0850jQuery.tex
%\input CH8/0830prototype.js.tex
%\input CH8/0840crossbrowser.tex
\fi