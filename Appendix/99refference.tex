% -*- coding: utf-8 -*-
\chapter{参考文献について}
%\setcounter{page}{1}
残念ながらSVGのまとまった解説がある日本語の本はないようです。
このテキストを書来始めたころに参考にした本は\cite{Cagle}と\cite{Campesato}です。
\ifSeminor
\else
また、途中からは\cite{Eisenberg}の図書も参考にしました。\fi
これらの本のリンク先は\href{http://www.amazon.co.jp}{amazon.co.jp}です。
平成18年4月現在で検索した書籍のページにジャンプします。

なお、SVGのフルの規格を調べるのであればやはり元の文書\cite{SVG11}を参考に
するしかありません。PDF版をダウンロードしておけば文書内にリンクが張って
あるので比較的簡単に該当箇所へジャンプできます(700ページ以上のあります)。

\cite{Cagle}の図書については次のような特徴があります。
\begin{itemize}
 \item 発行年次が2002年と少し古くSVG1.0に基づいた記述のようです。
 \item SVGの図形やテキストについての記述が詳しいです。
\ifSeminor
\else
 \item フィルターについても基本的なところが詳しくかかれています。本テキ
       ストを作成するに当たっては参考になった部分が多いです。
\fi
 \item インターラクティブなSVGのところもかなり詳しく書かれています。
\end{itemize}

\cite{Campesato}の図書については次のような特徴があります。
\begin{itemize}
 \item 似たようなSVGの例が多いので前半部は飛ばして読めます。。
 \item \JS を用いた部分の解説がかなり多いのでインターラクティブなSVGを作
       成したい人には参考になるでしょう。
\ifSeminor
\else
 \item その反面、フィルタの解説が少ないので物足りなく感じるかもしれませ
       ん。
       \fi
\end{itemize}
\ifSeminor
\else
\cite{Eisenberg}の図書については次のような特徴があります。
\begin{itemize}
 \item オーソドックスなSVGの解説書です。
 \item フィルターについては基本的なものだけです。
 \item フィルタの例については\cite{SVG11}のものと似ています。
 \item CSS の一覧表があります。
\end{itemize}
\fi
参考となるSVGのファイルや説明が日本語であるWebサイトは
検索するとたくさん見つかります。しかし、本書のようにDOMを用いてSVG文書を
操作する方法を説明しているサイトはほとんど見つかりません。
%\cite{hp}\cite{sec}\cite{hp2}などがあります。また、Adobe社のSVGゾーンに
%あるサンプルはかなり参考になります。

なお、今後HTMLも含めてXMLベースの文書を操作するには DOM が基本になります。
JavaScript の解説書でも最近のものはDOMを取り扱っています。
%今まで一通りの\JS でHTML文書を操作したことがある人は\cite{FirstJS}など
%で知識を見直しておくとよいでしょう。
%まだ日本語で\JS とDOMを同時に取り扱っている書籍はほとんどありませんが、
%外国語では\cite{DOMScripting}や
%\cite{DHTML}などがやさしく解説されています(この中にSVGに言及があればもっ
%と楽しいと思うのは著者だけでしょうか)。

JavaScript の解説書は非常にたくさんあります。なかでも\cite{JavaScript}は
言語の解説書として手元に置いておくとよいでしょう。また、姉妹書の
\cite{JavaScriptRef} は関数の仕様を調べるのに役に立ちます。

JavaScript は注意してコーディ
ングをすることが必要です。プログラミングスタイルも様々
ですが、大規模なアプリケーションをチームで組む必要があるときなどは
\cite{JavaScriptMentenable}を参考にしてください。また、より良いコーディ
ングスタイルを身に着けておきたい方は\cite{JavaScript68}がお勧めです。
また、使用を避けるべきJavaScriptの点については\cite{Crockford}がおすすめ
です。この本の記述はEcmaScriptのバージョン3に基づいているようです。現在
ECMAScriptのバージョンは2015年に出た6が最新版です\cite{ECMAScript2015}。
\ifSeminor
\else
バージョン5では配列に関するいろいろなメソッドやJSONの処理が標準で備わっ
ています。この部分が古くなっていることに注意してください。
\fi
\ifSeminor
\else
また、本文ではそれほど解説しませんで
したが Ajax によるWebアプリケーションは標準の技術になってし
まいました。
%このテキストでは\cite{Ajax}を参考に記述させてもらいました。
%今後Web2.0の時代に向けて中心的な
%技術になると思われます。このテキストでは\cite{Ajax}を参考に記述させても
%らいました。なお、Ajaxについては\cite{javascriptHacker}にも記述があります。
%が\cite{Ajax}のほうがエラー処理などの点でコードの質が高い気がします。
\fi
\ifSeminor
\else

2014年に策定されたHTML5はW3Cのサイト\cite{HTML5}その内容を見
ることができます最近のブラウザは HTML5 の機能をサポートしてい
ます。また、
iPhone や iPod Touch に搭載されている Safari もHTML5に対応してる
ばかりでなく、特有の機能であるマルチタッチを利用するAPIもあります
\footnote{iPhone Javascript マルチタッチ などのキーワードで検索すれば解
説しているサイトが見つかるでしょう。}。
%Wii に関しては\pageref{WiiSVG}ページを見てください。
\fi
\ifSeminor
\else
テキスト内で簡単に触れた PHP については\cite{learningphp,ProgPHP}などがあります。PHP
 は Web 時代後の言語なので、Web 関係の処理が簡単にできる機能が言語自体に
 備わっています。しかし、PHP の%能力としてAWK\cite{AWK} から始まる
 スクリプト系の言語としての面も丁寧に解説した書籍はほとんどありません。PHP
 の全貌を知る唯一の情報は PHP のホームページ\cite{PHPHome} です。
 \fi
 \INPUT{Appendix}{thebibliography}

