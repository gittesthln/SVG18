% -*- coding: utf-8 -*-
SVG に基づいた画像を作成する方法については次章で解説をします。
%
SVG は XML の構造を持ったベクター形式の画像を定義するフォーマットです。
したがって、SVG に基づいたファイルだけでは単なる文字の羅列にしか
見えず、これだけでは画像を表示することはできません。画像を表示するた
めにはソフトウェアが必要になります。このテキストの題名である Web
Graphics の観点からブラウザにより表示することを主に考えます。
最近のブラウザはSVGの画像の表示をサポートしています。
\iffalse
が、
SVGの機能の一つであるアニメーション(第\ref{ChapAnimation}章で説明します)
のうち色のアニメーションについてはサポートしていません。
色のアニメーションはCSSで実現できるので2017年度からはこの方法で説明をし
ます。
\fi

%\subsection{SVGが取り扱えるソフトウェア}
ブラウザ以外のいくつかのソフトウェアでもSVGファイルを取り扱うことができ
ます。

Adobe 社のドローソフト Illustlator は 簡単なSVGのファイルを表示したり、
結果をSVGファイルとして保存することが可能です
%\footnote{筆者がSVGの存在を
%知ったとき、別のベクター形式の画像ファイルからIllustratorを用いてSVGファ
%イルを作成しました。その内容を見てSVGでは比較的簡単に図形が描けることが
%わかったので授業でその内容を取り上げることにしたという経緯があります。}
。

%\href{http://www.imagemagick.org/script/index.php}{ImageMagick} はコマン
%ドラインから実行する画像処理ソフトです。SVGファイル
%を jpeg 形式のファイルに直すことができますが、SVGに対するサポートは不十分
%です。
