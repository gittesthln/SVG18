% -*- coding: utf-8 -*-
%\section{HTML5について}
2015年3月現在、代表的なブラウザは最新の\keyitem{HTML5}機能をインプリメントしています。
\subsection{HTML5までの道程}
W3C は文法上あい
まいであったHTML4の規格を厳密なXMLの形式に合う形にするために
\keyitem{XHTML}\cite{XHTML}を制定しました。これとは別にベンダー企業は
2004年に\keyitem{WHATWG}(Web Hypertext Application Technology Working Group)
を立ち上げ、XHTMLとは別の規格を検討し始めました。
2007年になってW3CはWHATWGと和解をし、WHATWGの提案を取り入れる形でHTML5の
仕様の検討が始まり、2014年10月に規格が正式なもの(Recommendation)になりました。

\subsection{HTML5 における新機能}
HTML5 では HTML 文書で定義される要素だけではなく、それに付随する概念も含
めます。HTML5における新機能としては次のようなものがあります。
\begin{itemize}
 \item 文書の構成をはっきりさせる要素が導入されています。文書のヘッダー
       部やフッター部を直接記述できます。
 \item フォントの体裁を記述する要素が非推奨となっています。これらの事項
       はCSSで指定することが推奨されています。
 \item SVG画像をインラインで含むことができます。

 \item \keyitem{WebStrage}

ローカルのコンピュータにそのWebページに関する情報を保存して、あとで利用
       できる手段を与えます。機能としてはCookieと同じ機能になりますが、
       Cookie がローカルに保存されたデータを一回サーバーに送り、サーバー
       がそのデータを見てWebページを作成するのに対し、WebStorage ではそ
       のデータがローカルの範囲で処理されている点が一番の違いです。

       Web ページの中には2度目に訪れた時に以前の情報を表示してくれるとこ
       ろがあります。これは Cookie に情報を保存しておき、再
       訪したとき、ブラウザが保存されたデータを送り、そのデータを用いてサーバーが
       ページを構成するという手法で実現しています。WebStorage を用い
       ると保存されたデータをサーバーに送ることなく同様のことが実現可能と
       なるのでクライアントとサーバーの間での情報の交換する
       量が減少します。また、WebStorage では保存できるデータの量
       が Cookie と比べて大幅に増加しています。
 \item \ElmH{canvas}

       インラインでの画像表示の方法として導入されました。画像のフォー
       マットはビットマップ方式で、\JS を用いて描きます。描かれた図形は
       オブジェクト化されないので、マウスのクリック位置にある図形はどれ
       かなどの判断は自分でプログラムする必要があります。

また、アニメーションをするためには途中の図形の形や位置などを自分で計算す
       る必要があり、複数の図形のアニメーションの同期も自分で管理する必
       要があります。

			 \iffalse
似たような形で図形を表示する
       \keyitem{Processing}\footnote{\url{http://processig.org}}という言
       語があります。図形の描き方の手順はほとんど同じです。
       実際にProcessing のプログラムをエミュレー
       トして\ElmH{canvas}に表示する
       \texttt{processing.js}\footnote{\url{http://processingjs.org}}というラ
       イブラリーも存在します。
			 \fi
\end{itemize}
このテキストでははじめは単独でSVGによる画像を描きますが、途中からはHTML
文書内でSVGの画像を表示し、それに対して構成要素を変えるプログラミングに
ついて解説します。