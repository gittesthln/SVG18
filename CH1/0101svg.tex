% -*- coding: utf-8 -*-
Web上での各種規格は \keyitem{World Wide Web  Consortium}(\keyitem{W3C})
と呼ばれる組織が中心になって制定しています。その
\Href{http://www.w3.org/standards/webdesign/}{Web Design and Applications}
(\Aref{図}{w3c-design}参照)には次のよう
に書かれています。
\ShowFig{0.9}{ht}{w3c-webdesign}{World Wide Web  Consortium の Web Design and
Application のページ(2017/2/27参照)}{w3c-design}
\begin{quotation}
Web Design and Applications involve the standards for building and
 Rendering Web pages, including HTML, CSS, SVG, device APIs, and other
 technologies for Web Applications (“WebApps”). This section also
 includes information on how to make pages accessible to people with
 disabilities (WCAG), to internationalize them, and make them work on
 mobile devices.
\end{quotation}

このページの Graphics をクリックすると W3C のグラフィックス関係の規格の
解説のページへ移動します(図\ref{w3c-graphics})。
%\clearpage

このページの下に SVG の簡単な説明があります。
\ShowFig{0.9}{p}{w3c-graphics}{W3C の SVG に関するトップページ
(2017/2/27参照)}{w3c-graphics}
\begin{quotation}
Scalable Vector Graphics (SVG) is like HTML for graphics. It is a markup
 language for describing all aspects of an image or Web application,
 from the geometry of shapes, to the styling of text and shapes, to
 animation, to multimedia presentations including video and audio. It is
 fully interactive, and includes a scriptable DOM as well as declarative
 animation (via the SMIL specification). It supports a wide range of
 visual features such as gradients, opacity, filters, clipping, and
 masking.
 
The use of SVG allows fully scalable, smooth, reusable graphics, from
 simple graphics to enhance HTML pages, to fully interactive chart and
 data visualization, to games, to standalone high-quality static
 images. SVG is natively supported by most modern browsers (with plugins
 to allow its use on all browsers), and is widely available on mobile
 devices and set-top boxes. All major vector graphics drawing tools
 import and export SVG, and they can also be generated through
 client-side or server-side scripting languages.
\end{quotation}

SVG の正式な規格書は図\ref{w3c-graphics}のページの右側にある CURRENT
STATUS の部分の SVG をクリックするとみることができます(図\ref{svg-rec})。
\ShowFig{0.9}{ht}{w3c-svg-recommendation}{SVG の規格のページ
(2017/2/27参照)}{svg-rec}
%\clearpage

画像を作成し、保存する形式を大きく分けるとビットマップ方式とベクター方式
があります。

ビットマップ方式は画像を画素(ピクセル)という単位に分け、その色
の情報で画像を表します。したがってはじめに決めた大きさで画像の解像度が決
まります。 Adobe社の
Photoshop や Windows に標準でついてくるペイントはビットマップ方式の画像
を作成するツールです。

これに対し、ベクター方式では線の開始位置と終了位置
(または長さと方向)、線の幅、色の情報などをそのまま持ちます。したがって
画像をいくら拡大しても画像が汚くなることはありません。ベクター方式のソフ
トウェアはドロー系のソフトウェアとも呼ばれます。代表的なものとしては
Adobe社の Illustrator があります。

SVG はその名称からもわかるようにベクター
形式の画像を定義するフォーマットのひとつです。

\begin{PreLearn}次の事柄について調べなさい。
\begin{enumerate}
 \item 画像の保存形式を調べ、それがビットマップ方式かベクター方式か調べ
       なさい。
 \item ビットマップ方式とベクター方式の画像形式の利点と難点をそれぞれ述
       べなさい。
\end{enumerate}
\end{PreLearn}
%\clearpage