% -*- coding: utf-8 -*-
\keyitem{XML}は eXtensible Markup Language の略です。
``Markup Language'' の部分を説明します。

ワープロなどで文書を作成するとき、ある部分は見出しなので字体を
ゴシックに変えるとい
うことをします。このように文書には内容の記述の部分とそれを表示するための
情報を含んだ部分があることになります。このように本来の記述以外の内容を
含んだ文書をハイパーテキストと呼びます。Microsoft Word のようなワープロ
ソフトが作成する文書もハイパーテキストです。このようなハイパーテキストを
表現するために文書の中にタグと呼ばれる記号を挿入したものが Markup
Language です。

Webのページは普通 \keyitem{html}(HyperText Markup Language) と呼ば
れる形式で記述されています。HTML文書は先頭が \ElmH{html}で最後が
\texttt{</html>} で終わっています。ここで現れた \ElmH{html} が
\keyitem{タグ}と呼ばれるものです。HTML文書ではこのタグが仕様で決まってい
てユーザが勝手に
タグを定義することができません。一方、XML では extensible という用語が示
すようにタグは自由に定義できます。したがって、XMLを用いることでいろいろ
な情報を構造化して記述することが可能になっています。XMLについてもっと知りた
い場合には文献\cite{IntroXML}などを参照してください。

