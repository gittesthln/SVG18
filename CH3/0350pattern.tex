% -*- coding: utf-8 -*-
\section{SVGパターン}
長方形などの内部を塗るための\AttribO{fill}には繰り返しのパターンを指定す
ることができます
\iffalse\footnote{\FF はSVGパターンをサポートしていません。}\fi 。

%パターンを用いると次のような錯視図形が簡単に描けます。

ヘルマン格子(図\ref{hermann-line})をパターンを利用して描くと
次のようになります。
%\newpage
\SVGListN{ヘルマン格子(パターンで描く)}{svg-hermann}{svg-hermann}
\begin{itemize}
 \item 内部が黒で、縁取りを白で塗る正方形を敷き詰める形でこの図
       形を描きます。
 \item 基本となる図形は\Elm{pattern}で定義します(\LineR{start}{end})。
        \Elm{pattern}はグラデーションのときと同じように\Elm{defs}内
	に記述します。
 \begin{itemize}
  \item \Elm{pattern}では後で参照するための属性\Attrib{id}とパターンの大
	きさを\AttribO{width}と\AttribO{height}で指定します(\Line{start})。
  \item \Line{coordinate}で\Elm{pattern}を塗る基準の座標系を属性
	\AttribO{patternUnits}を用いて指定していま
	す。グラデーションのときと同様に
	\AttribCVal{objectBoundingBox}{}と\AttribCVal{userSpaceOnUse}{}
	が指定できます。
  \item \Elm{pattern}内の図形は大きさが\texttt{50}の正方形で縁取りの幅を
	$8$にした正方形です(\LineR{rects}{rects})。
 \end{itemize}
 \item このパターンで内部を塗る長方形は
       \Line{rectfillede}で定義されていま
       す。\AttribO{fill}に\Elm{pattern}の属性\Attrib{id}の値を
       指定します。この場合には\AttribVal{url}{(\#Hermann)}と
       \AttribCVal{url}{()}を付けます。
\end{itemize}
\ProbwithFigSol{savigny}{0.4}{ht}{ザビニの錯視}{svg-savigny}
{\OIIdxM{ザヴィニ}\cite[132ページ]{Ninio}とよばれます。両方のパターンを囲む正方形の大
きさは同じなのですがパターンの向きにより大きさが異なる別の長方形に
見えます。この図を描きなさい。}

パターン内の図形としてはいくつかの図形を組み合わせ
てもかまいません。図\ref{pattern-example}は別の
グラデーションで塗った複数の正方形を組み合わせてパターンを構成しています。
%
\ShowFig{0.35}{ht}{pattern}
   {二つの線形グラデーションを市松模様に並べた例}
{pattern-example}
\SVGListN{二つの線形グラデーションを市松模様に並べた例}
   {patterndef1}{svg-pattern}
\begin{itemize}
 \item  \Line{grad1S}から\Line{grad1E}は上から下へ向
	かう線形グラデーションを、 \Line{grad2S}から\Line{grad2E}は左から右へ向
	かう線形グラデーションを定義しています。
 \item 辺の長さが$60$の正方形を二つこれらのグラデーションを使って
       塗ります。
 \item この二つの正方形を市松模様に並べたパターンを\Line{patternS}から
       \Line{patternE}で定義します。
\begin{itemize}
 \item パターンの大きさは正方形を縦横二つずつ並べたものなので大きさは
       $2\times60=120$です。したがって、\AttribO{width}{}と
       \AttribO{height}{}の値はそれぞれ$120$です(\Line{patternS})
 \item グラデーションで塗られた2種類の正方形を\Elm{use}を用いて対角
       線上に並べます(\Line{use1}から\Line{use4})。
 \item \Elm{use}のなかに配置する位置\AttribO{x}や\AttribO{y}を指定してい
       ます。
\end{itemize}
 \item \Line{rect}から始まる長方形をこのパターンで塗りつ
       ぶします。ひとつのパターンの大きさが$120\times120$なのでこのパターンが横
       に3回, 縦に2回繰り返されます。
\end{itemize}
\iffalse
\begin{Problem}\upshape
 傾いたグラデーションを市松模様に並べた図形を作成しなさい。なお、
各パターンは長方形の辺に平行でかまいません。
\end{Problem}
\fi

%\clearpage
\ProbwithFigSol{hermann-twinkle}{0.35}{ht}{輝くヘルマン格子}{svg-hermann-twinkle}
{ないはずの点がよりはっきり見えるようになる
\OIIdxN{輝くヘルマン格子}{180}{}です。
%\par
この図では正方形の間の隙間は少し明るめの灰色で、交差部分には
白で塗られた円を描いています。これもパターンを用いてこの図を作成しなさい。}
%
%
\ProbwithSol%
{munsterberg-pattern1}{カフェウォール錯視--パターンで作成}
{svg-munsterberg-pattern}
{\keysubE{カフェウォール}{錯視}{}錯視(図\ref{munsterberg})をパターンを
用いて作図しなさい。}\IndexSet{カフェウォール錯視}{}{}{}{}{}

\ProbwithFigSol{morgan-twist}{0.35}{ht}{モーガンのねじれひも}{svg-morgan-twist}
{\OIIdxN{モーガンのねじれひも}{76}{}とよばれる錯視図形です。これを作
成しなさい。}
%\ifFull\newpage\fi
パターンを塗りつぶす図形は長方形でなくてもかまいません。

図\ref{oouchi}は大内元が作成した錯視図形の内部のパターンを
簡略化した図形です(\cite[74ページ図7.5]{Ninio})。\footnote{\cite[75ペー
ジ]{Ouchi}も参照のこと。この本にはほかにも面白い錯視図形が載っています。}
中央の円がページの画面から浮き上がって揺れているように見えます。
%片目で見ると立体感が増すように見えます。
%これは、円をパターンで塗りつぶしています。

\ShowFig{0.35}{ht}{oouchi}{\OIIdxB{浮動する円}(簡略版)}{oouchi}
%\ifFull\newpage\fi
\SVGListN{浮動する円}{svg-oouchi-rev}{svg-oouchi}
\begin{itemize}
 \item \LineR{patternS}{patternE}でパターンを定義しています。
 \item パターンは全体を白で塗りつぶし長方形(\Line{rectW})の
       上に対角線上に二つ並んだ黒く塗りつぶされた長方形(\Line{path})からなります。
 \item \LineR{CS}{CE}で外側の円をこのパターンで塗りつぶしています。
 \item この円の上に同じパターンで塗りつぶされた小さな円(\Line{C2})を
       $90^{\circ}$ 回転(\Line{rotate})して描
       いています。
\end{itemize}
%\SolSVGFile{問題}{prob-oouchi}{浮動する円}{svg-oouchi}
\begin{Problem}\upshape
 図\ref{polygon5}の内部をパターンで塗りつぶしなさい。パターンの開始がど
 こから始まっているかを調べること。
\end{Problem}
なお、\Elm{pattern}の属性にはパターンの配置を変形させる
\AttribO{patternTransform}があります。この値は図形を移動させる属性
\Showattrib{transform}の値と同じものが書けます。
\begin{Problem}\upshape\label{prob-pattern-rotate}
    図\ref{pattern-rotate}は
図\ref{pattern-example}のグラデーションのパターンを回転さ
せた図形を内部に持つ長方形です。\Elm{pattern}に属性
 \AttribO{patternTransform}を用いてこの図を作成しなさい。
    \ShowFig{0.35}{ht}{pattern-rotate}
{図\ref{pattern-example}のグラデーションのパターンを回転させる}
{pattern-rotate} %patterndef-rotate-rev.svg
\end{Problem}
%\vspace{-3em}