% -*- coding: utf-8 -*-
\section{\JS とは}
\JS というと\HTML 文書の中で取り扱われ、主にブラウザで利用される言語と思っ
ている人が多いと思います。しかし、\JS の言語仕様のもととなるものは
\keyitem{ECMA}が定めている \keyitem{ecmascript} と呼ばれるものです。一般
のプログラミング言語と同様に、言語の構成要素が定義されています。ブラウザ
などで使用される場合には、この言語を基に、ブラウザ上で定義されたオブジェ
クトやそれを操作する方法を用いてプログラムを組むことになります。現に、
\JS を網羅的に解説している\cite{JavaScript}では第I部が「コア\JS」第II部
が「クライアントサイド\JS」と分けられています。最近では「サーバーサイド\JS」
と呼ばれるものも登場しています。

これらの違いは、プログラミングを記述するための手段として\JS を利用するこ
とで、利用する環境に応じて別のライブラリーが用意されているために、いろい
ろな環境で利用が可能となっています。これはC言語で簡単なプログラムを記述
する場合には複雑なライブラリーは必要としませんが、ウィンドウを開いて実行
されるプログラムにはウィンドウを開くためのライブラリーが必要になってくる
との似ています。

とはいっても、コアの\JS だけを使用してプログラムを組むことを解説している
参考書は皆無といえます。このテキストではコアの\JS によるプログラミ
ングを解説をします。とはいっても、コアの\JS を実行する環境はやはりブラウ
ザとなります。ブラウザ上でどのようにコアの\JS を実行し、デバッグする方法
を示しながら、\JS のプログラミング言語としての特徴をつかんでください。