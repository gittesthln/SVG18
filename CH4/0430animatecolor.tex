\section{色を変える(\Elm{animateColor})}
属性にアニメーションを付けるためには\Elm{animateColor}を使用します。
しかし、このアニメーションは後で解説する\Elm{animate}でも実現できるので
将来の規格ではなくなるとの記述があります
\footnote{\href{https://www.w3.org/TR/SVG/animate.html#AnimateColorElement}
の最後の部分}。

図\ref{animation-color}は円の塗り(\AttribO{fill})と縁取り
(\AttribO{stroke})にそれぞれアニメーションをつけています。

\ShowFigs{0.22}{ht}
{{animation-color-begin}{animation-color-2}{animation-color-end}}
{色のアニメーション(開始時--左--、途中(中央)、終了時--右--)}
{animation-color}
\SVGListN{色のアニメーション}
{svg-animation-color}{svg-animation-color}

色の名前は CSS で定義されているので \AttribA{attributeType} の属性値は
\texttt{CSS} となります。\vspace{-1.\baselineskip}
