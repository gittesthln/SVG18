\section{道のりに沿ったアニメーション(\Elm{animateMotion})}
指定した道のりに沿って動くアニメーションでは
\Elm{animateMotion}を用います。道程は属性\AttribO{path} で指定します。
\ShowFig{0.4}{ht}{motion-along-path}{道のりに沿って動くアニメーション}
{motion-along-path}
\SVGListN{道のりに沿ったアニメーション}{svg-motion-along-path}
   {svg-motion-along-path}
\begin{itemize}
 \item \LineR{pathS}{pathE}でアニメーションで動くパスを定義しています。このパ
       スは動きがわかるように表示にも使われています(\Lines{path1}{path2})。
 \item \LineR{item1S}{item1E}ではアニメーションで動く左側の長方形を定義
       しています。
       この長方形の内部を塗る\AttribO{fill}が\AttribCVal{currentColor}{}で
       あることに注意してください。これは上位の環境で定義されている色で
       塗ることを意味します。ここでは\Line{g1}にある\Elm{g}の属性
       \AttribO{color}の値(\AttribCVal{red})が利用されます。
 \item \LineR{anim1S}{anim1E}でこの長方形にアニメーショ
       ンをつけています。\Elm{defs}のなかで定義された道のりを参照するため
       に\Elm{mpath}を用いています。
 \item 右側の長方形についても同様のアニメーションが付きます
       (\LineR{item2S}{item2E})。このアニメーションには属性\AttribA{rotate}に
       \AttribAVal{auto}{}を指定しているので道のりに接するように長方形が
       回転しながら移動します。\AttribAVal{reverse-auto}{}という値も指定
       できます。
\end{itemize}
\begin{Problem}\upshape
 リスト\ref{svg-motion-along-path}においてアニメーションの属性
 \AttribA{rotate}に\AttribAVal{reverse-auto}{}を設定したときの動きを確認
 しなさい。
\end{Problem}
